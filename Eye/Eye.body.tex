\title{Perception Requires Shapeliness}
\author{Jonathan D. Lettvin}
\begin{document}
\maketitle
\begin{abstract}

Transforming a visual scene to optic nerve signal is explored.

Photons from RGB monitors are diffracted and refracted over a retina.
Opposed wavelength variance by diffraction and refraction
cause Airy disk size to be wavelength-invariant.
Refraction causes wavelength-variant differential radial displacement.

Photoreceptors decay to equilibrium state with flux.
Photoreceptors generate signal on flux changes.
Horizontal cells average Photoreceptor state.
Horizontal cells generate signal on change of average.
Difference between Photoreceptor and Horizontal cell signals is
detected by dendritic spine TGS units (transient gradient samplers).

Dendritic spines are distributed over paraboloid extensions of bipolar cells.
Coincident detection of transient gradients constitutes a feature detection.
Centrifugal bipolar cells select section of paraboloid.
Pupillary reflex couples to section selection.

Character of human optical input is demonstrated.
After image in photoreceptor response is demonstrated.
Shapes of bipolar cell have consequence.
Hyperacuity is demonstrated.

\end{abstract}

\section{Introduction}
Eye is a model of how a scene projects onto a retina,
how a retina converts converts projections into coherent signal,
and how feedback refines the conversion.

\section{Discussion}
\bibliographystyle{plain}
\bibliography{Eye}
\appendix

\pagebreak
\section{How seen movement appears in the frog's optic nerve}\label{appendix:A}\cite{LettvinSeenMovement}
Transcribed from a reprint

Federation Proceedings Volume 18 Number 1 March 1959 Pages 393 and 354.
H. Maturana, J. Y. Lettvin, W. H. Pitts, W. S. McCulloch
Research Laboratory of Electronics, M.I.T. Cambridge, MA
Transcribed in its entirety.

\subsection{Part I (page 393)}
The receptive field of a single optic nerve fiber
(plotted by the on and off responses to small fixed spots)
is often divisible into concentric cones.
This suggests that the response of the fiber to a moving spot
may be polar with respect to a reference point in the receptive field.
Movement is indeed polarly encoded and there exists
at least the following four types of fibers whose rate of firing depends on
the centrifugal component of a movement
with respect to some point internal to the receptive field
(centripetal and tangential movements never cause discharge):
Some fibers have wide receptive fields and low sensitivity.
Of these some prefer the moving object darker than background,
other prefer it lighter.
A second group has constricted fields and high sensitivity.
A third set shows a directional heavy weighting of the response.
A fourth kind has annular fields.
A fifth variety measures inversely
the average intensity of illumination in a region.
Its maximum rate is in the dark.

\subsection{Part II (page 354)}
The coding of movement described in Part I
suggests that the frog's eye is designed
(at least for land operation)
to abstract the vector and size of a moving object
and extrapolate the path.
Because our evidence implies that there exists a coordinate system
built into the retina and that the coding allows coordinates and velocity
to arise from general operations on the whole output of the optic nerve,
we propose some alternative guesses to account for
Sperry's results on dislocated eyes.
We do not propose that his notion of specific reconnection is wrong
but that it is not necessary.
We also present the law by which there is a point-to-point correspondence
from receptors to optic nerve, vis.,
if an object is moved within the visual field
in a circular path of any diameter,
the only fibers that show no response at any time
are those that have the centers of their receptive fields
at the center of the circle described.
